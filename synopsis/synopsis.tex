\documentclass[12pt,a4paper,twocolumn]{article}
\usepackage{times} % times font
\usepackage{mathptmx} % times font in maths
\usepackage[top=0.65in, bottom=0.65in, left=0.65in, right=0.65in]{geometry}
\usepackage{multirow} %in tables
\usepackage{caption} % in tables
\pagenumbering{gobble}
\newcommand{\HRule}{\rule{\linewidth}{0.5mm}}

\usepackage[pdftex]{graphicx}
\usepackage{lipsum}
\usepackage{amsmath}

% \usepackage{hyperref}
% \usepackage{subfigure}
% \usepackage{indentfirst} % indent frst paragraph of section
% \usepackage[usenames,dvipsnames]{color}
\newcommand{\ts}{\textsuperscript}

\begin{document}

\twocolumn[
\begin{@twocolumnfalse}
\begin{center}
	\begin{large}
	{\HRule \\[0.2cm]}
	\textsc{Instance-wise thresholding methods in multi-label classification}
	{\HRule \\[0.3cm]}
	\end{large}

	\begin{minipage}{ 0.44\textwidth }
		\begin{flushleft}
			\textit{Author:}\\
			Kacper \textbf{Sokol}
		\end{flushleft}
	\end{minipage}
	\begin{minipage}{ 0.44\textwidth }
		\begin{flushright}
			{\textit{Supervisors:}\\
			Peter \textbf{Flach}, Meelis \textbf{Kull}\\[0.3cm]}
		\end{flushright}
	\end{minipage}
\end{center}
\end{@twocolumnfalse}
]

\section*{\texttt{Review}}
This work will cover instance-wise thresholding on multi-label ranking. To be more precise, we will examine choosing a threshold based on particular data instance on ordered, according to their relevance, list of labels.\\
We are going to evaluate performance of such task measured on: \emph{yeast}, \emph{emotions}, \emph{scene} and \emph{delicious} or \emph{reuters} datasets.\\
To train the classifier and rank corresponding test instances we will use: \emph{MEKA}, \emph{Mulan} and independent implementation of \emph{Probabilistic Classifiers Chains}.\\
As evaluation measures we plan to employ example based measures: \emph{recall}, \emph{precision}, \emph{$F_\beta$}, \emph{accuracy}, and \emph{hamming loss}; label based measures: \emph{macro-averaging} and \emph{micro-averaging}; ranking measures: \emph{one-error}, \emph{coverage}, \emph{ranking loss} and \emph{average precision}; finally we will also use \emph{hierarchical-loss}.\\

\section*{\texttt{Multi-label ranking}}
Multi-label ranking is used to order the labels according to their relevance given particular instance. Then the thresholding algorithm is applied to allow selecting relevant number of labels per instance.\\
For our experiment we will use the following ranking and scoring approaches:
\begin{enumerate}
\item Rankings:
	\begin{itemize}
	\item Multi-label perceptron. %*
	\item Ranking by pairwise comparison. %*
	\item Calibrated label ranking. %*
	\item Ad- aBoost.MR. %*
	\end{itemize}
\item Scores:
	\begin{itemize}
	\item Random k-Labelsets a.k.a.\ RAkEL---label power-set. %*
	\item Ensembles of pruned set---label power-set.
	\item Ensembles of classifier chains---pruned sets i.e.\ extension of label power-set reducing cardinality.
	\end{itemize}
\end{enumerate}

For the approaches which based on binary or multi-class solutions we aim to test them with: \emph{k-NN}, \emph{SVM} and \emph{decision trees}.

\section*{\texttt{Data driven thresholding}}
We aim at applying the following data dependent thresholding methods:
\begin{enumerate}
\item MetaLabeler---regression from the feature vector to the number of labels per instance based on:
	\begin{itemize} % 05670068.pdf
	\item Original input space.
	\item Score vectors.
	\item Sorted score vectors.
	\end{itemize}
	and:
	\begin{itemize} % 05670068.pdf
	\item Regression.
	\item Multi class classification.
	\end{itemize}

\item Artificial label for pairwise comparisons---PCC, EPCC, etc.
\item Artificial label to extract threshold value.
\item Score per label\ts{*}.
\item Threshold prediction---based on $t(x)$ minimizing:
	$$
	| \lambda_j \in Y : s_j(x) \leq t(x) | + | \lambda_j \in \Delta \text{\textbackslash{}} Y : s_j(x) \geq t(x) |
	$$
\end{enumerate}

\section*{\texttt{Empirical comparison}}
Finally we aim to ``compare and contrast'' acquired results according to aforementioned criteria. Our goal is also to indicate winning thresholding algorithm if possible and discover any dependencies that may arise between type of dataset, manner of fixing the threshold, and ranking approach.

% \begin{figure}[htbp]
% \centering
% \includegraphics[width=0.5\textwidth]{figures/figure4b.png}
% % \begin{tiny}
% \caption{Simulation of two interconnected neurons with inhibitory synapses.\label{fig:part4b}}
% % \end{tiny}
% \vspace{0.2cm}
% \end{figure}

\end{document}

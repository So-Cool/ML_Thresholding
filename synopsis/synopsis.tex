\documentclass[12pt,a4paper,twocolumn]{article}
\usepackage{times} % times font
\usepackage{mathptmx} % times font in maths
\usepackage[top=0.7in, bottom=0.7in, left=0.7in, right=0.7in]{geometry}
\usepackage{multirow} %in tables
\usepackage{caption} % in tables
\pagenumbering{gobble}
\newcommand{\HRule}{\rule{\linewidth}{0.5mm}}

\usepackage[pdftex]{graphicx}
\usepackage{lipsum}
\usepackage{amsmath}

% \usepackage{hyperref}
% \usepackage{subfigure}
% \usepackage{indentfirst} % indent frst paragraph of section
% \usepackage[usenames,dvipsnames]{color}
% \newcommand{\ts}{\textsuperscript}

\begin{document}

\twocolumn[
\begin{@twocolumnfalse}
\begin{center}
	\begin{large}
	{\HRule \\[0.2cm]}
	\textsc{Instance-wise thresholding methods in multi-label classification}
	{\HRule \\[0.3cm]}
	\end{large}

	\begin{minipage}{ 0.44\textwidth }
		\begin{flushleft}
			\textit{Author:}\\
			Kacper \textbf{Sokol}
		\end{flushleft}
	\end{minipage}
	\begin{minipage}{ 0.44\textwidth }
		\begin{flushright}
			{\textit{Supervisors:}\\
			Peter \textbf{Flach}\\
			Meelis \textbf{Kull}\\[0.3cm]}
		\end{flushright}
	\end{minipage}
\end{center}
\end{@twocolumnfalse}
]

\section*{\texttt{Review}}
This work will cover instance-wise thresholding on multi-label ranking. We are going to evaluate performance of such task measured on: \emph{yeast}, \emph{emotions}, \emph{scene} and \emph{reuters} datasets. To train the classifier and rank corresponding test instances we will use: \emph{MEKA}, \emph{Mulan} and independent implementation of \emph{Probabilistic Classifiers Chains}.\\
As evaluation measures we plan to employ: \emph{recall}, \emph{precision}, \emph{$F_\beta$}, \emph{accuracy}, and \emph{hamming loss}.\\

\section*{\texttt{Multi-label ranking}}
Multi-label ranking is used to order the labels according to their relevance given particular instance. Then the thresholding algorithm is applied to allow selecting relevant number of labels per instance.\\
For our experiment we will use the following ranking approaches:
\begin{enumerate}
\item Rankings:
	\begin{itemize}
	\item Multi-class, multi-label perceptron.
	\item Ranking by pairwise comparison.
	\end{itemize}
\item Scores:
	\begin{itemize}
	\item Random k-Labelsets (RAkEL)---(Label power-set).
	\item Ensembles of pruned set---(Label power-set).
	\item Ensembles of classifier chains---(Pruned sets---extension of Label power-set reducing cardinality).
	\end{itemize}
\end{enumerate}


\section*{\texttt{Data driven thresholding}}
\begin{enumerate}
\item MetaLabeler---Regression from the feature vector to the number of labels per instance.
	\begin{itemize} % 05670068.pdf
	\item Original input space.
	\item Score vectors.
	\item Sorted score vectors.
	\end{itemize}

	or:

	\begin{itemize} % 05670068.pdf
	\item Regression.
	\item Multi class classification.
	\end{itemize}

\item Artificial label with pairwise comparisons---PCC, EPCC, etc.
\item Artificial label to extract thresholding number.
\item Score per label.
\item Threshold prediction---based on $t(x)$ minimizing:
	$$
	| \lambda_j \in Y : s_j(x) \leq t(x) | + | \lambda_j \in \Delta \text{\textbackslash{}} Y : s_j(x) \geq t(x) |
	$$
\end{enumerate}

\section*{\texttt{Empirical comparison}}
Finally we aim to ``compare and contrast'' acquired results. Our goal is also to indicate winning thresholding algorithm if possible and discover any dependencies that may arise between type of dataset, manner of fixing the threshold, and particular method of ranking.

% \begin{figure}[htbp]
% \centering
% \includegraphics[width=0.5\textwidth]{figures/figure4b.png}
% % \begin{tiny}
% \caption{Simulation of two interconnected neurons with inhibitory synapses.\label{fig:part4b}}
% % \end{tiny}
% \vspace{0.2cm}
% \end{figure}

\end{document}
